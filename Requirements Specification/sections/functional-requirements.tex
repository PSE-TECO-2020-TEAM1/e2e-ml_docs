\section{Functional Requirements}

\subsection{Main Functions}
These functions must be implemented in order to fulfill the core criteria.

\subsubsection{Web Client}
The web client supports both desktop and mobile modes. The functionality that will be displayed is determined by the device information of the browser.

\paragraph{Desktop Web Client}

% improve on this with cross referencing, need to learn latex tho, later
\def\twodigits#1{%
  \ifnum#1<10 0\fi
  \number#1}

\begin{enumerate}[{label = \textbf{/F{\protect\twodigits{\arabic{enumi}}}0/}, leftmargin = *}]
    \item \label{/F010/} Show a welcome page with the registration and login panel
    \item List workspaces of the logged in user
    \item Allow creating workspaces given the name and sensors to be used
    \item List the available sensors for the recordings
    \item Allow renaming workspaces
    \item Show a workspace panel when a workspace is selected
    \item Allow creating labels for the actions to be recorded on the workspace
    \item List labels with their sample count
    \item Allow renaming labels
    \item Allow deleting labels which in turn deletes the data samples with the selected label
    \item Display a link and the QR code with the same link embedded to be used for recording data
    \item Display the collected data samples chronologically on the workspace panel
    \item Display the metadata recording and allow selecting the relevant timeframes of the sample
    \item Allow relabeling and deleting samples
    \item Allow selecting the possible model training options on the management panel % needs to be more specific % after ml workshop
    \item Request training and deploying a model according to the selected options
    \item List the trained models
    \item Display the used parameters and data samples of the selected model
    \item Display the performance metrics of the selected model
    \item Display a link and the QR code with the same link embedded to be used for classifying data
    % \item Request the processing of the data according to the selected options
\end{enumerate}

\paragraph{Mobile Web Client}
\begin{enumerate}[resume*]
    \item Show a configuration page with available labels
    \item Allow configuring the countdown duration until the recording starts
    \item Allow configuring the recording duration
    \item Allow configuring the sampling rate of the sensors
    \item Show a button to initiate the recording
    \item Show a countdown page
    \item Display the current configuration on the countdown page
    \item Show a recording page
    \item Display the sensor data in real-time as curve graphs
    \item Show a recording completed page
    \item Allow discarding the last recording
    \item Send the sensor data to the server
    \item Allow another recording with the same configurations
    \item Allow editing configurations for the next recording
\end{enumerate}

\subsubsection{Server API}
\begin{enumerate}[resume*]
    \item Serve authentication services
    \item Serve workspace information for a user
    \item Serve data set and recording link for a workspace
    \item Create, rename and delete labels on a workspace
    \item Serve label information to the mobile client
    \item Accept data from the mobile client
    \item Initiate the configured model training
    \item Serve the model information on a workspace
    \item Rename and delete models on a workspace
\end{enumerate}

% \subsubsection{Data Processing}
% \begin{enumerate}[resume*]
%     \item \textbf{TBD, after workshop}
% \end{enumerate}

\subsection{Extending Functions}
\begin{enumerate}[resume*]
    \item Allow listing samples by label on the workspace panel
    \item Give non-visual feedback (e.g play a sound) to user if the identification is successful
\end{enumerate}