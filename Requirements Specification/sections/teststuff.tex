\section{Test Cases and Test Scenarios}
\subsection{Test Cases}
Each functionality is tested by at least one of the following test cases. The test cases are divided into two parts to clarify the distinction between main and extending functions. The written test scenarios aim to provide the testers with a structurally defined order of events that will unfold during the ordinary use of application.
\subsubsection{Main Test Cases}
These test cases are necessary for the web-tool to work properly and succeed in serving
basic requests. Each test case represents a basic action user can take.

\paragraph{Desktop Web Client}

\begin{enumerate}[{label = \textbf{/T{\protect\twodigits{\arabic{enumi}}}0/}, leftmargin = *}]
    \item Access web page (\ref{itm:welcome-page}) %\hyperref[welcome_page]{/F010/}
    \begin{itemize}
        \item \textbf{/T011/} via Google Chrome,
        \item \textbf{/T012/} via Mozilla Firefox,
        \item \textbf{/T013/} via Internet Explorer 6.
    \end{itemize}
    \item Sign up with (\ref{itm:welcome-page} \ref{itm:registration-panel})
    \begin{itemize}
        \item \textbf{/T021/} new credentials,
        \item \textbf{/T022/} existing credentials.
    \end{itemize}
    \item Login (\ref{itm:welcome-page}, \ref{itm:authentication})
    \begin{itemize}
        \item /T031/ with existing credentials,
        \item /T032/ with invalid credentials.
    \end{itemize}
    \item Create a new \gls{workspace} with a name, sampling rate value and required \glspl{sensor}. (\ref{itm:create-workspace}, \ref{itm:cr-list-sensors}, F540)
    \item Open an already existing \gls{workspace}. (\ref{itm:workspaces-overview}, \ref{itm:workspace-panel}, \ref{itm:display-samples})
    \item Rename the \gls{workspace}. (\ref{itm:rename-workspace})
    \item Open the labels overview. (\ref{itm:labels-overview})
    \begin{itemize}
        \item /T071/ Create new label. (\ref{itm:create-label})
        \item /T072/ Rename label. (\ref{itm:rename-label})
        \item /T073/ Delete label. (\ref{itm:delete-label})
    \end{itemize}
    \item Initiate data collection
    \begin{itemize}
        \item /T081/ via using the link (\ref{itm:workspace-link}, \ref{itm:workspace-copy-link})
        \item /T082/ via scanning the \gls{QR code}. (\ref{itm:workspace-qr})
    \end{itemize}
    \item Open the sample overview. (\ref{itm:visualize-sample}, \ref{itm:metadata})
    \item Choose another graph type to visualize data. (\ref{itm:change-graph})
    \item Set the relevant time frame of the selected \gls{data sample} on the graph. (\ref{itm:set-timeframe})
    \item Relabel the selected sample. (\ref{itm:relabel})
    \item Delete the selected sample. (\ref{itm:delete-sample})
    \item Select the desired data imputation options. (\ref{itm:imputation})
    \item Select the desired \gls{feature extraction} options. (\ref{itm:feature-extraction})
    \item Select the normalization options. (\ref{itm:normalization})
    \item Select the desired machine learning \glslink{machine learning model}{model}. (\ref{itm:model-hyperparameters})
    \item Set the hyperparameters of the \glslink{machine learning model}{model}. (\ref{itm:model-hyperparameters})
    \item Train the \glslink{machine learning model}{model}. (\ref{itm:train})
    \item Open the \glslink{machine learning model}{model}s overview. (\ref{itm:models-over-list}, \ref{itm:disp-miniqr-classify})
    \item Open the overview of a specific \glslink{machine learning model}{model}. (\ref{itm:models-over-over}, \ref{itm:models-over-1}, \ref{itm:models-over-2})
    \item Initiate data \gls{classification} (\ref{itm:disp-miniqr-classify}, \ref{itm:disp-link-classify})
    \begin{itemize}
        \item /T221/ via using the link, (\ref{itm:disp-copy-classify})
        \item /T222/ via scanning the \gls{QR code}. (\ref{itm:disp-qr-classify})
    \end{itemize} 
\end{enumerate}

\paragraph{Mobile Web Client}
\begin{enumerate}[resume*]
    \item Choose a \gls{label} for the data that will be recorded. (\ref{itm:conf-rec-label}, \ref{itm:conf-rec-nosensor})
    \item Set the countdown to start of recording. (\ref{itm:conf-rec-count})
    \item Set the recording duration. (\ref{itm:conf-rec-dur})
    \item Start recording data. (\ref{itm:init-rec-button}, \ref{itm:init-rec-countdown}, \ref{itm:init-rec-realtime}, \ref{itm:init-rec-completed})
    \item Discard recorded data. (\ref{itm:discard-record})
    \item Record another data after one recording completes. (\ref{itm:next-record1}, \ref{itm:next-record1})
    \item Record data to classify. (\ref{itm:show-classification}, \ref{itm:start-listening}, \ref{itm:display-sensor}, \ref{itm:diplay-classified})
    \item Stop the \gls{classification}. (\ref{itm:stop-recording}, \ref{itm:display-recording-chrono})
    \item Restart the \gls{classification}. (\ref{itm:restart-recording})
\end{enumerate}

\subsubsection{Extending Test Cases}
These test cases are not necessary for the application to work properly, but vital for testing the extending functions. 
\begin{enumerate}[resume*]
    \item Toggle "Stay Signed In" checkbox. (\ref{itm:stay-signed})
    \item Solve \gls{CAPTCHA}. (\ref{itm:CAPTCHA})
    \item Check if a data collecting device is connected on the desktop client. (\ref{itm:active-sign})
    \item Filter samples by label. (\ref{itm:filter-by-label})
    \item Add a description to the label. (\ref{itm:label-description})
    \item Set a sound to play in case of a \gls{classification} on the desktop web client. (\ref{itm:sound-feedback})
    \item Change the language of the desktop web client. (\ref{itm:language})
    \item Select the \gls{magnetometer} \gls{sensor}. (\ref{itm:magnetometer})
    \item Select "Forgot Password". (\ref{itm:forgot-password})
\end{enumerate}

\subsection{Test Scenarios}
The following test scenarios are composed of the aforementioned test cases. Starred test scenarios indicate that extending functionalities are also tested in the scenario.
\subsubsection{Test Scenario 1 - First Time User}
A user who is new to machine learning applications opens the webpage on their computer and registers to the service. After logging in, the user collects data using the default configuration with their phone and trains a machine learning \glslink{machine learning model}{model}. The user then deploys this \glslink{machine learning model}{model} and uses the deployed \glslink{machine learning model}{model}s' unique \gls{QR code} to identify new data they record in real-time. Lastly, the user stops the \gls{classification} and examines the classified actions.
\begin{enumerate}
    \item /T010/ Access web page.
    \item /T021/ Sign up with new credentials.
    \item /T031/ Login with existing credentials.
    \item /T040/ Create a new \gls{workspace} with a name, sampling rate value and required \glspl{sensor}.
    \item /T070/ Open the labels overview.
    \item /T071/ Create new label.
    \item /T081/ Initiate data collection via scanning the \gls{QR code}.
    \item /T230/ Choose a \gls{label} for the data that will be recorded.
    \item /T260/ Start recording data.
    \item /T280/ Record another data after one recording completes.
    \item /T190/ Train the \glslink{machine learning model}{model}.
    \item /T200/ Open the \glslink{machine learning model}{model}s overview.
    \item /T221/ Initiate data \gls{classification} via scanning the \gls{QR code}.
    \item /T290/ Record data to classify.
    \item /T300/ Stop the \gls{classification}.
\end{enumerate}
\subsubsection{Test Scenario 2 - Invalid credentials}
User tries to login with invalid credentials first and manages to login after a few unsuccessful attempts.
\begin{enumerate}
    \item /T010/ Access web page.
    \item /T032/ Login with invalid credentials.
    \item /T032/ Login with invalid credentials.
    \item /T031/ Login with existing credentials.
\end{enumerate}
\subsubsection{Test Scenario 3 - Unsupported Desktop Client}
User opens the website on an unsupported desktop browser and is welcomed with an error message.
\begin{enumerate}
    \item /T013/ Access web page via Internet Explorer 6.
\end{enumerate} 
\subsubsection{Test Scenario 4 - Unsupported Mobile Client}
User scans the \gls{QR code} they received to begin recording data. It turns out their device doesn't support the \Gls{sensor} API and the website notifies the user about it.
\begin{enumerate}
    \item /T092/ Initiate date collection via scanning the \gls{QR code}.
\end{enumerate}
\subsubsection{Test Scenario 5 - Forgotten Password}
User tries to login with an invalid username and/or password. After a few attempts user requests a new password using the "Forgot Password" functionality and successfully manages to login using their new password.
\begin{enumerate}
    \item /T010/ Access web page.
    \item /T032/ Login with invalid credentials.
    \item /T032/ Login with invalid credentials.
    \item /T400/ Select "Forgot Password".
    \item /T031/ Login with existing credentials.
\end{enumerate} 
\subsubsection{Test Scenario 6 - Data Deletion} 
User logs in and opens their previously set \gls{workspace}. User finds some \glspl{data sample} and labels to be improper and deletes them.
\begin{enumerate}
    \item /T010/ Access web page.
    \item /T031/ Login with existing credentials.
    \item /T050/ Open an already existing \gls{workspace}.
    \item /T130/ Delete the selected sample.
    \item /T070/ Open the labels overview.
    \item /T072/ Delete label.
\end{enumerate}
\subsubsection{Test Scenario 7 - Interruption during Recording}
User uses the link that was provided previously to start collecting data. After configuring the options the user starts the recording. Whilst recording, the user receives a phone call. Recording continues during the call. The user then discards the recording and start a new recording. 
\begin{enumerate}
    \item /T010/ Access web page.
    \item /T031/ Login with existing credentials.
    \item /T050/ Open an already existing \gls{workspace}.
    \item /T081/ Initiate data collection via using the link.
    \item /T230/ Choose a \gls{label} for the data that will be recorded.
    \item /T240/ Set the countdown to start of recording.
    \item /T250/ Set the recording duration.
    \item /T260/ Start recording data.
    \item /T270/ Discard recorded data.
    \item /T280/ Record another data after one recording completes.
    \item /T260/ Start recording data.
\end{enumerate}
\subsubsection{Test Scenario 8 - Registration with Already Existing Credentials}
User opens the website on their desktop device and tries to register with an already existing username.
\begin{enumerate}
    \item /T010/ Access web page.
    \item /T022/ Sign up with existing credentials.
\end{enumerate}
\subsubsection{Test Scenario 9 - Multi-language Support*}
User opens the website and changes the language option to German. 
\begin{enumerate}
    \item /T010/ Access web page.
    \item /T380/ Change the language of the desktop web client.
\end{enumerate}
\subsubsection{Test Scenario 10 - Extensive Use of Main Functionalities}
User opens the website and creates a new account. 
\begin{enumerate}
    \item /T010/ Access web page. 
    \item /T021/ Sign up with new credentials.
    \item /T032/ Login with invalid credentials.
    \item /T031/ Login with existing credentials.
    \item /T040/ Create a new \gls{workspace} with a name, sampling rate value and required \glspl{sensor}.
    \item /T060/ Rename the \gls{workspace}.
    \item /T070/ Open the labels overview.
    \item /T071/ Create new label.
    \item /T072/ Rename label.
    \item /T071/ Create new label.
    \item /T073/ Delete label.
    \item /T080/ Initiate data collection via scanning the \gls{QR code}.
    \item /T230/ Choose a \gls{label} for the data that will be recorded.
    \item /T240/ Set the countdown to start of recording.
    \item /T250/ Set the recording duration.
    \item /T260/ Start recording data.
    \item /T270/ Discard recorded data.
    \item /T280/ Record another data after one recording completes.
    \item /T230/ Choose a \gls{label} for the data that will be recorded.
    \item /T240/ Set the countdown to start of recording.
    \item /T250/ Set the recording duration.
    \item /T260/ Start recording data.
    \item /T090/ Open the sample overview.
    \item /T100/ Choose another graph type to visualize data.
    \item /T120/ Relabel the selected sample.
    \item /T110/ Set the relevant time frame of the selected \gls{data sample} on the graph.
    \item /T130/ Delete the selected sample.
    \item /T140/ Select the desired data imputation options.
    \item /T150/ Select the desired \gls{feature extraction} options.
    \item /T160/ Select the normalization options.
    \item /T170/ Select the desired machine learning \glslink{machine learning model}{model}.
    \item /T180/ Set the hyperparameters of the \glslink{machine learning model}{model}.
    \item /T190/ Train the \glslink{machine learning model}{model}.
    \item /T200/ Open the \glslink{machine learning model}{model}s overview. 
    \item /T210/ Open the overview of a specific \glslink{machine learning model}{model}.
    \item /T082/ Initiate data via scanning the \gls{QR code}.
    \item /T290/ Record data to classify. 
    \item /T300/ Stop the \gls{classification}.
    \item /T310/ Restart the \gls{classification}.
    \item /T300/ Stop the \gls{classification}.
\end{enumerate}

\subsubsection{Test Scenario 11 - Extensive Use of All Functionalities*}
\begin{enumerate}
    \item 
\end{enumerate}
\subsubsection{Test Scenario 12 - Brute-force Attempt*}
User with malicious intents tries to hack an account through brute-forcing. After 5 unsuccessful login attempts, they are presented with a \gls{CAPTCHA} to solve.
\begin{enumerate}
    \item /T010/ Access web page.
    \item /T032/ Login with invalid credentials.
    \item /T032/ Login with invalid credentials.
    \item /T032/ Login with invalid credentials.
    \item /T032/ Login with invalid credentials.
    \item /T032/ Login with invalid credentials.
    \item /T330/ Solve \gls{CAPTCHA}.
\end{enumerate}
\subsubsection{Test Scenario 13 - Multiple/Simultaneous Data Collection and Classification}
Two mobile clients will first collect data for one workspace at the same time. Then they will classify actions at the same time.
\begin{enumerate}
    \item /T081/ Initiate data collection via using the link. (1)
    \item /T082/ Initiate data collection via scanning the \gls{QR code}. (2)
    \item /T230/ Choose a \gls{label} for the data that will be recorded. (1)
    \item /T230/ Choose a \gls{label} for the data that will be recorded. (2)
    \item /T240/ Set the countdown to start of recording. (1)
    \item /T260/ Start recording data. (2)
    \item /T250/ Set the recording duration. (1)
    \item /T260/ Start recording data. (2)
    \item /T221/ Initiate data \gls{classification} via using the link. (1)
    \item /T222/ Initiate data \gls{classification} via scanning the QR code. (2)
    \item /T290/ Record data to classify. (1)
    \item /T290/ Record data to classify. (2)
    \item /T300/ Stop the \gls{classification}. (1)
    \item /T300/ Stop the \gls{classification}. (2)
\end{enumerate}