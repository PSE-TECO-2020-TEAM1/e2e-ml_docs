\section{Think-Aloud Tests}
\subsection{Think-Aloud Test 1}
\subsubsection{Participant}
Age: 25
\newline
Profession: Electrical Engineering Student
\newline
Machine Learning Experience: None
\newline

\subsubsection{Participant Transcript}
"I am trying to connect with non-existant [sic] credentials which didn't work. So we need to sign up first. Signing up is very smooth, few personal data required. Username, password and email, which was very quick. After creating a new account, I am able to connect. 
The design is very simple and right away you can see what you should do next. Clicking on create new workspace displays a window where you can choose a name for the workspace and the corresponding sensors and sample [sic] rates. I choose as sensor Gyroscope and Accelerometer both with a sample [sic] rate of 50. And then click create. Now I can select it with the chosen name in the workspace page. When clicking on it, you can edit its name, add new labels, edit models and delete the workspace. I created a vertical label, horizontal and one for the shaking. I need to enter the name of the label each time and click on create. Labels are ordered from 1 to 3 and you can delete the labels you want. We have 0 samples, so we should collect some date [sic]. Then when clicking on collect data a QR-code shows up. I think that for someone new to machine learning the message should ask to copy the link in a device having the chosen sensors. I don't have QR scanner in my phone, So copying the link to my phone browser takes some time. After copying the link I opened it and I am asked to choose one of the labels I created. I will start with the vertical label. I can edit the parameters of the record: countdown and duration and put them to 10 each. And I clicked record. I put my device in a vertical position. And I can see in the chart the collected data with 6 inputs 3 for each sensor. By choosing a point on the chart, you can see values of the inputs in the chosen point of time. If I'm not happy with the recorded sample, I can restart the recording with the same configuration or choose another configuration. For the time I will accept the sample so I click on send sample. When going back to the computer browser I can't see the sample but loading is written. After reconnecting and going back to the workspace the sample is there. And it shows that the label vertical has 1 sample but the others none. Now I will create a sample for the horizontal label. I choose the same configuration and run the recording. We can see the countdown and directly after it starts collecting data for 10 sec. We can see the difference between the vertical chart and the horizontal one. I send this sample again. It shows an error saying that the label does not exist. I tried to delete it and create the label again. This time I am able to send data. I collect data one more time for vertical and one for shaking. Now all samples are on my workspace with 2 samples for vertical one for shaking and one for horizontal. Now I create a new model. I can see for the accelerometer and gyroscope in all 3 x,y,z axes: imputations, features and normalizer. I don't have a big idea about these characteristics so I choose randomly. On the right side I can choose a name for my model, its classifier and its hyperparameters including the window size, sliding step .. I will fill them randomly too. Now I can press train model. The steps of the training shows up and become green one by one. When going back to my workspace, I can see my model. I can delete it or go to a QR link for identification. Which I think should be classify since identification is not that clear. I again copy the link and opened it in my mobile browser. I can see a chart similar to the recording chart. And above we can see the program is trying to predict the action that the smartphone is performing. For the most of the time the real-time prediction is right. But sometimes when a lot of labels occur at the same time it can lead to a delay. All in all I think the goal of the experience is reached but there are some bugs especially when collecting a sample."

\subsubsection{Participant SUS}
\begin{table}[h]
\resizebox{\columnwidth}{!}{%
\begin{tabular}{!{\VRule}c!{\VRule}c!{\VRule}c!{\VRule}c!{\VRule}c!{\VRule}c!{\VRule}}
 \hline
    & 1 &2 & 3 & 4 &5                \\
\hline
I think I would like to use this system frequently &  & & x &  &  \\
\hline
I found the system unecessarily complex & X & &  &  & \\
\hline
I thought the system was easy to use &  & &  &  &X              \\
\hline
I think that I would need the support of a technical person to be able to use this system &  &X &  &  &   \\
\hline
I found the various functions in this system were well integrated &  & &  &X  &   \\
\hline
I thought there was too much inconsistency in this system &  & X& &  &  \\
\hline
I would imagine that most people would learn to use this system very quickly &  & &  &  &X \\
\hline
I found the system very cumbersome to use &  &X &  &  &              \\
\hline
I felt very confident using this system &  & & X &  &   \\
\hline
I need to learn a lot of things before I could get going with this system &  & &X  &  &   \\
\hline
\end{tabular}%
}
\end{table}
\subsubsection{Feedback}
\begin{itemize}
  \item For someone with no experience in machine learning imputations, features ,normalizer, hyperparameters, sliding window should be more explained.
  \newline
  \item In rare cases a loading problem occurs when sending a sample in the desktop client and the user needs to reconnect to his account.
  \newline
  \item In rare cases when trying to send a sample an error saying that the label does not exist occurs. Recreating the label fixes the problem.
  \newline
\end{itemize}